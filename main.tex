\documentclass[journal,12pt,twocolumn]{IEEEtran}

\usepackage{setspace}
\usepackage{gensymb}

\singlespacing


\usepackage[cmex10]{amsmath}

\usepackage{amsthm}

\usepackage{mathrsfs}
\usepackage{txfonts}
\usepackage{stfloats}
\usepackage{bm}
\usepackage{cite}
\usepackage{cases}
\usepackage{subfig}

\usepackage{longtable}
\usepackage{multirow}

\usepackage{enumitem}
\usepackage{mathtools}
\usepackage{steinmetz}
\usepackage{tikz}
\usepackage{circuitikz}
\usepackage{verbatim}
\usepackage{tfrupee}
\usepackage[breaklinks=true]{hyperref}
\usepackage{graphicx}
\usepackage{tkz-euclide}
\usepackage{float}

\usetikzlibrary{calc,math}
\usepackage{listings}
\usepackage{color} %%
\usepackage{array} %%
\usepackage{longtable} %%
\usepackage{calc} %%
\usepackage{multirow} %%
\usepackage{hhline} %%
\usepackage{ifthen} %%
\usepackage{lscape}
\usepackage{multicol}
\usepackage{chngcntr}

\DeclareMathOperator*{\Res}{Res}

\renewcommand\thesection{\arabic{section}}
\renewcommand\thesubsection{\thesection.\arabic{subsection}}
\renewcommand\thesubsubsection{\thesubsection.\arabic{subsubsection}}

\renewcommand\thesectiondis{\arabic{section}}
\renewcommand\thesubsectiondis{\thesectiondis.\arabic{subsection}}
\renewcommand\thesubsubsectiondis{\thesubsectiondis.\arabic{subsubsection}}


\hyphenation{op-tical net-works semi-conduc-tor}
\def\inputGnumericTable{} %%

\lstset{
%language=C,
frame=single,
breaklines=true,
columns=fullflexible
}
\begin{document}


\newtheorem{theorem}{Theorem}[section]
\newtheorem{problem}{Problem}
\newtheorem{proposition}{Proposition}[section]
\newtheorem{lemma}{Lemma}[section]
\newtheorem{corollary}[theorem]{Corollary}
\newtheorem{example}{Example}[section]
\newtheorem{definition}[problem]{Definition}

\newcommand{\BEQA}{\begin{eqnarray}}
\newcommand{\EEQA}{\end{eqnarray}}
\newcommand{\define}{\stackrel{\triangle}{=}}
\bibliographystyle{IEEEtran}
\providecommand{\mbf}{\mathbf}
\providecommand{\pr}[1]{\ensuremath{\Pr\left(#1\right)}}
\providecommand{\qfunc}[1]{\ensuremath{Q\left(#1\right)}}
\providecommand{\sbrak}[1]{\ensuremath{{}\left[#1\right]}}
\providecommand{\lsbrak}[1]{\ensuremath{{}\left[#1\right.}}
\providecommand{\rsbrak}[1]{\ensuremath{{}\left.#1\right]}}
\providecommand{\brak}[1]{\ensuremath{\left(#1\right)}}
\providecommand{\lbrak}[1]{\ensuremath{\left(#1\right.}}
\providecommand{\rbrak}[1]{\ensuremath{\left.#1\right)}}
\providecommand{\cbrak}[1]{\ensuremath{\left\{#1\right\}}}
\providecommand{\lcbrak}[1]{\ensuremath{\left\{#1\right.}}
\providecommand{\rcbrak}[1]{\ensuremath{\left.#1\right\}}}
\theoremstyle{remark}
\newtheorem{rem}{Remark}
\newcommand{\sgn}{\mathop{\mathrm{sgn}}}
\providecommand{\abs}[1]{\left\vert#1\right\vert}
\providecommand{\res}[1]{\Res\displaylimits_{#1}}
\providecommand{\norm}[1]{\left\lVert#1\right\rVert}
%\providecommand{\norm}[1]{\lVert#1\rVert}
\providecommand{\mtx}[1]{\mathbf{#1}}
\providecommand{\mean}[1]{E\left[ #1 \right]}
\providecommand{\fourier}{\overset{\mathcal{F}}{ \rightleftharpoons}}
%\providecommand{\hilbert}{\overset{\mathcal{H}}{ \rightleftharpoons}}
\providecommand{\system}{\overset{\mathcal{H}}{ \longleftrightarrow}}
%\newcommand{\solution}[2]{\textbf{Solution:}{#1}}
\newcommand{\solution}{\noindent \textbf{Solution: }}
\newcommand{\cosec}{\,\text{cosec}\,}
\providecommand{\dec}[2]{\ensuremath{\overset{#1}{\underset{#2}{\gtrless}}}}
\newcommand{\myvec}[1]{\ensuremath{\begin{pmatrix}#1\end{pmatrix}}}
\newcommand{\mydet}[1]{\ensuremath{\begin{vmatrix}#1\end{vmatrix}}}
\numberwithin{equation}{subsection}
\makeatletter
\@addtoreset{figure}{problem}
\makeatother
\let\StandardTheFigure\thefigure
\let\vec\mathbf
\renewcommand{\thefigure}{\theproblem}
\def\putbox#1#2#3{\makebox[0in][l]{\makebox[#1][l]{}\raisebox{\baselineskip}[0in][0in]{\raisebox{#2}[0in][0in]{#3}}}}
\def\rightbox#1{\makebox[0in][r]{#1}}
\def\centbox#1{\makebox[0in]{#1}}
\def\topbox#1{\raisebox{-\baselineskip}[0in][0in]{#1}}
\def\midbox#1{\raisebox{-0.5\baselineskip}[0in][0in]{#1}}
\vspace{3cm}
\title{Assignment-05}
\author{Suyog Tangade\\MD/2020/710}
\maketitle
\newpage
\bigskip
\renewcommand{\thefigure}{\theenumi}
\renewcommand{\thetable}{\theenumi}
Download all python codes from
\begin{lstlisting}
https://github.com/suyogtangade/Assignment5.git
\end{lstlisting}
%
and latex-tikz codes from
%
\begin{lstlisting}
https://github.com/suyogtangade/Assignment5.git
\end{lstlisting}
%
Question taken from
\begin{lstlisting}
https://github.com/gadepall/ncert/blob/main/linalg/quadratic_forms/gvv_ncert_quadratic_forms.pdf-Q.no.2.31
\end{lstlisting}
%
\section{quadratic forms.pdf-Q.no.2.31}
Find the equation of the hyperbola with foci $\myvec{0\\ \pm 12}$ and length of the latus rectum 36.
\section{Solution}
The equation of a conic with directrix $\vec{n}^{\top}\vec{x} = c$, eccentricity $e$ and focus $\vec{F}$ is given by 
\begin{align}
    \label{2/31/eq:conic_quad_form}
    \vec{x}^{\top}\vec{V}\vec{x}+2\vec{u}^{\top}\vec{x}+f=0
    \end{align}
The eccentricity of the conic  is given by
\begin{align}
e= \sqrt{1-\frac{\lambda_1}{\lambda_2}}
\label{2/31/eq:e}
\end{align}
\begin{definition}[Latus rectum]
The latus rectum of a conic section is the chord (line segment) that passes through the focus, is perpendicular to the major axis and has both endpoints on the curve.
\end{definition}
For $\abs{\vec{V}} \ne 0$, the lengths of semi-major and semi-minor axes of the conic  are 
\begin{align} 
\sqrt{\frac{\vec{u}^{\top}\vec{V}^{-1}\vec{u} -f}{\lambda_1}}, 
\sqrt{\bigg | \frac{f-\vec{u}^{\top}\vec{V}^{-1}\vec{u}}{\lambda_2}\bigg | }
\label{2/31/eq:ab}
\end{align} 
% For $\abs{\vec{V}} \ne 0$ the foci of the conic in \eqref{2/31/eq:conic_quad_form} are given by
% \begin{align}
%   \vec{F} &=\vec{c}\pm\brak{\sqrt{\frac{(\vec{u}^T\vec{V}^{-1}\vec{u}-f)(\lambda_2-\lambda_1)}{\lambda_1\lambda_2}}}\vec{p_1} \label{2/31/eq:foci}
% \end{align}
The equation latus rectum of the conic in \eqref{2/31/eq:conic_quad_form} is given by
\begin{align}
    \vec{n}^{\top}\brak{\vec{x}-\vec{F}} = 0
    \label{2/31/eq:lreq}
\end{align}
For $\abs{\vec{V}} \ne 0$, the length of latus rectum (LLR) of the conic in \eqref{2/31/eq:conic_quad_form} is given by 
\begin{align} 
LLR=\dfrac{2\bigg | \dfrac{f-\vec{u}^{\top}\vec{V}^{-1}\vec{u}}{\lambda_2}\bigg | }{{\sqrt{\dfrac{\vec{u}^{\top}\vec{V}^{-1}\vec{u} -f}{\lambda_1}}}}
\label{2/31/eq:LLR}
\end{align} 
\begin{proof}
Using \eqref{2/31/eq:ab}, we can write
\begin{align}
  \vec{F} &=\vec{c}\pm\brak{\sqrt{\frac{(\vec{u}^T\vec{V}^{-1}\vec{u}-f)(\lambda_2-\lambda_1)}{\lambda_1\lambda_2}}}\vec{p_1} \label{2/31/eq:foci}
\end{align}
Given, length of latus rectum is 36 and focii are $\myvec{0\\ \pm12}$. Let us consider $\myvec{0\\ 12}$ for solving the problem.
\begin{align}
    \vec{F} =\myvec{0\\ 12}\Rightarrow\norm{\vec{F}}=12
\end{align}
Let $\vec{u}^{\top}\vec{V}^{-1}\vec{u}-f=\alpha$. From \eqref{2/31/eq:ab},\eqref{2/31/eq:e},\eqref{2/31/eq:LLR}
\begin{align}
    \sqrt{\dfrac{\alpha}{\lambda_1}}\sqrt{1-\frac{\lambda_1}{\lambda_2}}=12\label{2/31/eq:one}\\
    \dfrac{2\brak{\dfrac{-\alpha}{\lambda_2}}}{\sqrt{\dfrac{\alpha}{\lambda_1}}}=36\label{2/31/eq:two}
\end{align}
Dividing \eqref{2/31/eq:one} by \eqref{2/31/eq:two} gives
\begin{align}
    \dfrac{\lambda_1}{\lambda_2}&=-3\\
    \Rightarrow e&=2\label{2/31/eq:q}\\
    \Rightarrow \sqrt{\frac{\alpha}{\lambda_1}}&=6\label{2/31/eq:w}
\end{align}
The associated directrix is perpendicular to the y-axis and passes through the point
\begin{align}
\myvec{0\\\sqrt{\dfrac{\alpha}{e^2\lambda_1}}}=\myvec{0\\3}
\end{align}
Hence, its equation is
\begin{align}
    \myvec{0&1}\brak{\vec{x}-\myvec{0\\3}} &= 0\\
    \Rightarrow \myvec{0&1}\vec{x} &= 3
\end{align}
Comparing it with $\vec{n}^{\top}\vec{x} = c$
\begin{align}
    \vec{n} = \myvec{0\\1}, c = 3\Rightarrow \norm{\vec{n}} = 1
\end{align}
Calculating $\vec{V}, \vec{u}$ and $f$,
\begin{align}
    \vec{V}&=1^2\myvec{1&0\\0&1} - 2^2\myvec{0\\1}\myvec{0&1}\\
    &=\myvec{1&0\\0&1}-\myvec{0&0\\0&4}=\myvec{1&0\\0&-3}\\
    \vec{u}&= 3(2^2)\myvec{0\\1} - 1^2\myvec{0\\12}=\myvec{0\\0}\\
    f &= 1^2(12^2) - 3^2(2^2)= 108
\end{align}
Hence, the required equation is
\begin{align}
    \vec{x}^{\top}\myvec{1&0\\0&-3}\vec{x}+108=0
\end{align}
Also, from \eqref{2/31/eq:lreq}, the equations of latus rectum is
\begin{align}
    \myvec{0&1}\brak{\vec{x}-\myvec{0\\12}} &= 0\\
    \Rightarrow \myvec{0&1}\vec{x} &= 12
\end{align}
Similarly, the equations of directrix and latus rectum associated with $\myvec{0\\ -12}$ are given by
\begin{align}
    \myvec{0&1}\vec{x} &= -3\\
    \myvec{0&1}\vec{x} &= -12
\end{align}
The plot of the hyperbola is given in the fig \ref{fig:1}
\numberwithin{figure}{section}
\begin{figure}[H]
\centering
\includegraphics[width= \columnwidth]{Hyperbola.png}
\caption{Hyperbola} \label{fig:1}
\end{figure}

\end{document}

